\section{Exercise 4}
We have the following pde given
\begin{align}
u_t = u_{xxx} +6uu_x, \quad u=u(x,t) \label{eq:kdv}
\end{align}
which is known as the Korteweg-de Vries equation.
\paragraph{(a)}\label{ch:4a}
We want to find a symmetry transformation of the dependent and independent variables
\begin{align}
S:(t,x,u)\rightarrow(T,X,U) = (at,bt,cu),
\end{align}
with $a,b,c$ nonzero constants, such that \cref{eq:kdv} takes the form
\begin{align}
U_T = U_{XXX} +6UU_X, \quad U=U(X,T).\label{eq:transformed_kdv}
\end{align}
Before starting the calculations, we note the chain rule of partial derivatives 
\begin{align}
\pder{f(x(t))}{t} = \pder{x(t)}{t}\pder{f(x)}{x}.\label{eq:chain_rule}
\end{align}
We calculate the individual terms in \cref{eq:transformed_kdv}
\begin{align}
U_T(X,T)&=\pder{}{T}U=c\pder{}{T}u(x,t) \stackrel{\cref{eq:chain_rule}}{=} c \underbrace{\pder{t}{T}}_{1/a}\pder{u(x,t)}{t}=\frac ca u_t,\\
\intertext{In the same manner we calculate}
U_X&= c\frac{\partial x}{\partial X} \frac{\partial}{\partial x} u(x,t)=\frac{c}{b} u_{x},\\
U_{XXX}&=c \frac{\partial^3 x}{\partial X^3} \frac{\partial^3}{\partial x^3} u(x,t)=\frac{c}{b^3} u_{xxx}.
\end{align}
Plugging this into \cref{eq:transformed_kdv} we get
\begin{flalign}
 &&\frac ca u_t &= \frac{c}{b^3} u_{xxx} +6\frac{c^2}{b}uu_x,&&\\
\iff &&   u_t &= \frac{a}{b^3} u_{xxx} +6\frac{c}{b}uu_x.&&
\end{flalign}
So we get the conditions
\begin{flalign}
&&\frac{a}{b^3}=1, &\qquad \frac{ac}{b}=1,&&\\
\iff&& a=b^3,& \qquad c=\frac{1}{b^2}.&&
\end{flalign}
We choose $b=\epsilon$ with $\epsilon$ non zero and find
\begin{align}
X = \epsilon x, \quad T = \epsilon^3 t,\quad U = \frac{1}{\epsilon^2} u. \label{eq:sym_transf}
\end{align}
Now we assume, that $u(x,t) = f(x,t)$ is a solution to \cref{eq:kdv}. The found symmetries leave the equation invariant. So we can construct a second solution
\begin{align}
u(x,t) = \epsilon^2 f(X,T) =  \epsilon^2 f(\epsilon x, \epsilon^3 t).
\end{align}

\paragraph{(b)}
We propose a solution of the type 
\begin{align}
u(x,t)=f(\xi),\quad \xi= x-vt.
\end{align}
We insert this ansatz into\cref{eq:kdv}. With the chain rule \cref{eq:chain_rule} we calculate the individual terms
\begin{align}
u_t &= \partial_\xi u \underbrace{\partial_t \xi}_{-v}=-v \partial_\xi u,\\
u_x &= \partial_\xi u \partial_x \xi = \partial_\xi u ,\\
\Rightarrow u _{xxx}&= \partial_{\xi}^3 u.
\end{align}
The equation becomes 
\begin{align}
-v u_\xi = u_{\xi\xi\xi} + 6 u u_\xi.
\end{align}
Since $\xi$ is the only variable, we have an ODE, we will use $(\cdot)'$ as notation for derivative to make clear, we are dealing with an ODE
\begin{align}
-v u' = u''' +6 uu'.\label{eq:4b_ode}
\end{align}
We take the integral
\begin{align}
\int -v u' \dd\xi &= \int u''' +3(u^2)' \dd\xi\\
\Rightarrow v u+u''+3u^2 &= A.
\end{align}
With $A$ being some integration constant.  We multiply by $u'$ and integrate again
\begin{align}
\int vuu' +u'u'' + 3u'u^2 \dd\xi&= \int Au' \dd\xi\\
\Rightarrow \frac 12 vu^2 + \frac 12 (u')^2  + u^3&= Au +B
\end{align}
The Weierstrass $\mathscr{P}$-function solves equations of the form
\begin{align}
(\mathscr{P}')^2 = 4\mathscr{P}^3-g_1\mathscr{P}-g_2
\end{align}
We have to make a substitution, so we get rid of the quadradic term in $\mathscr{P}$.
We propose $u(\xi)=-2y(\xi)-\frac v6$. Inserting, gives (addition of cubic and quadratic term done with Wolframalpha)
\begin{align}
-8y^3+\frac{v^2y}{2}+\frac{v^3}{108} +2(y')^2&=A\left(-2y-\frac v6\right) +B\\
\iff (y')^2 &= 4y^3-\underbrace{\left(\frac{v^2}{4}+A\right)}_{g_1}y - \underbrace{\left(\frac{v^3}{216} +\frac{Av}{12} - \frac B2\right)}_{g_2}
\end{align}
So the solution to \cref{eq:kdv} is given by
\begin{align}
u(x,t) &= -2\mathscr{P}(\xi,g_1,g_2) - \frac v6, \label{eq:sol_4b}\\
\xi = x-vt,& \quad g_1=\frac{v^2}{4}+A, \quad g_2=\frac{v^3}{216} +\frac{Av}{12} - \frac B2.
\end{align}
\paragraph{(c)}
We know, from exercise (a) that \cref{eq:sol_4b} will also be solved by
\begin{align}
u(x,t)=\epsilon^2\left(-2\mathscr{P}(\epsilon\left( x-\epsilon^2v t\right), g_1,g_2)-\frac{v}{6}\right).
\end{align}
For larger $\epsilon$ the amplitude of $u$ will clearly be larger. But at the same time, the time evolution will be faster, because we have an additional $\epsilon^2$ in front of $t$, compared to $x$. Or we could say, that we absorb the $\epsilon^2$ into a new velocity $\tilde{v}=v\epsilon^2$ which will of course be larger for larger $\epsilon$. So the wave propagates faster.\\
In a linear equation this effect would not appear since a change in amplitude would always be present in each term of the component and thus simply factor out and not influence the solution. 
\paragraph{(d)}
We  can deduce with the symmetry transformation of \cref{eq:sym_transf}, that the following quantities are invariant under transformation
\begin{align}
X^2U&=\epsilon^2x^2\frac{1}{\epsilon^2} u = x^2u,\\
X^3T^{-1}&=\epsilon^3x^3(\epsilon^3t)^{-1}=x^3t^{-1}.
\end{align}
In contrast to exercise (c), we propose a solution of the form
\begin{align}
u(x,t) = \frac{1}{x^2}f(\xi),\quad \xi=x^3/t.
\end{align}
With a little help from your favourite computer algebra system, we calculate
\begin{align}
\partial_x u&=\frac 3t f'\left(\xi\right)-\frac{2}{x^3}f\left(\xi\right),\\
\partial_t u&=-\frac{x}{t^2}f'\left(\xi\right),\\
\partial^3_x u&=27 \frac{x^4}{t^3} f^{(3)}\left(\xi\right)+24\frac{1}{tx^2}f'\left(\xi\right)-24\frac{1}{x^5}f\left(\xi\right).
\end{align}
Inserting this into \cref{eq:kdv} gives us 
\begin{align}
-\frac{x}{t^2} f'(\xi)= 27\frac{x^4f'''(\xi)}{t^3} + 24 \frac{f'(\xi)}{tx^2} - 24 \frac{f(\xi)}{x^5} +6f(\xi)(\frac 3t f'(\xi)-\frac{2}{x^3}f(\xi)).
\end{align}
We multiply the equation with $x^5$ and rewrite everything in terms of $\xi$ and regrouping the terms
\begin{align}
0= 27 \xi^3 f''' + (24\xi+\xi^2) f' - 24 f + 9\xi(f^2)' - 12f^2.\label{eq:f_eq_d}
\end{align} 
Now we make the ansatz $f(\xi)=a$, with $a$ being a constant, we get
\begin{align}
0=-24a-12a^2
\rightarrow a=0 \vee a= -2.
\end{align}
We now want to find a rational solution, whose denominator and numerator are quadratic polynomials and $f(0)=0$. The most general form of this ansatz is given by
\begin{align}
f(\xi)=\frac{a\xi^2+b\xi}{c\xi^2+d\xi+e},\label{eq:ansatz_polynomial}\\
a\neq 0, c\neq 0, e\neq0.
\end{align}
We know that $f(\xi)$ satisfies \cref{eq:f_eq_d} for every $\xi\in\R$. So we pick 5 arbitrary points pairwise different points for $\xi$ and thus get a system of equations which we can solve for the coefficients $a,b,c,d,e$. When inserting these into  \cref{eq:ansatz_polynomial} we have to check if this solution solves \cref{eq:f_eq_d} for $\xi\in\R$. Luckily, we do not have to do this by hand, since Mathematica has a function "SolveAlways[]", which does this for us. We find
\begin{align}
\{c \rightarrow -(d/24), e \rightarrow -6 d, a \rightarrow d/4, b \rightarrow 6 d\}.
\end{align}
So we get the solution
\begin{align}
f(\xi) = \frac{\frac 14\xi^2+6\xi}{-\frac{1}{24}\xi^2+\xi-6}.
\end{align}
Or if we rewrite it in terms of $u(x,t)$
\begin{align}
u(x,t)=\frac{1}{x^2}\frac{\frac 14(x^3/t)^2+6x^3/t}{-\frac{1}{24}(x^3/t)^2+x^3/t-6} =\frac{\frac 14(x^4/t^2)+6x/t}{-\frac{1}{24}(x^8/t^2)+x^5/t-6x^2}.
\end{align}
\paragraph{(f)}
In one dimension we can write a conservation law as
\begin{align}
\int_a^b\rho(x,t)_t\dd x=-\int_a^b j(x,t)_x\dd x
\end{align}
When assuming $\rho$ and $\rho_x$ to be continuous in time, we can use Leibniz rule of integration on the left hand side, while integrating on the right
\begin{align}
\der{}{t}\int_a^b\rho(x,t) \dd x= j(a,t)-j(b,t).
\end{align}
If \[\lim_{x\to\pm\infty} j (x,t)= 0\] for all times, we find
\begin{align}
\der{}{t}\int_\R \rho \dd x = 0
\end{align}
to be conserved.
In the following we will assume $\lim_{x\to\pm\infty}u(x,t)=0$, $\lim_{x\to\pm\infty}u_x(x,t)=0$ and $\lim_{x\to\pm\infty}u_{xx}(x,t)=0$, since the total energy of the fluid $\int_\R u^2\dd x$ could otherwise become infinite.ß\\
For showing the first property, we integrate \cref{eq:kdv} in space
\begin{align}
\int_\R u_t \,\dd x&= \int_\R u_{xxx} +6uu_x\,\dd x\\
\iff \der{}{t} \int_\R u \,\dd x & = \int_R (\underbrace{u_{xx} + 3 u^2}_{j(x,t)})_x\, \dd x.
\end{align} 
Since we assume $u$ and its spatial derivatives to vanish at infinity, we have \[\lim_{x\to\pm\infty} j (x,t)= 0\] and thus
\begin{align}
\der{}{t} \int_\R u \dd x = 0.
\end{align}
For showing the second conservation, we multiply \cref{eq:kdv} by $2u$ and integrate in space
\begin{align}
\int_\R2 uu_t\,\dd x & = \int_\R 2uu_{xxx} +12u^2u_x \, \dd x\\
\iff\der{}{t}\int_\R u^2\, \dd x&= \int_R (2uu_{xx}-u_x^2)_x + (4u^3)_x\,\dd x
\end{align}
So we find $j(x,t)=2uu_{xx}-u_x^2 + 4u^3$ vanishing at spatial infinity and thus 
\begin{align}
\der{}{t}\int_\R u^2\, \dd x= 0.
\end{align}
The last one is a bit more tricky, we use the operation $3u^2-u_x\partial_x$ on the equation and integrate in space 
\begin{align}
\int_\R3u_t u^2 - u_{tx}u_x\dd x +\int_\R(3u^2-u_x\partial_x)(u_{xxx} +6uu_x)\dd x =0.
\end{align}
The first term can be directly calculated, for the second term we use our favourite computer algebra system to verify that we can rewrite as follows
\begin{align}
\der{}{t}\int_\R u^3 -\frac 12 u_x^2 \dd x -\frac 12\int_\R(\underbrace{6u^2u_{xx}+u_{xx}-12uu_x^2-2u_{xxx}u_x+9u^4}_{j(x,t)})_x\dd x = 0.
\end{align}
So, again $j(x,t)$ vanishes under our assumptions at spatial infinity and thus
\begin{align}
\der{}{t}\int_\R u^3 -\frac 12 u_x^2 \dd x  = 0.
\end{align}
\paragraph{(g)}
For this paragraph we simply integrate the invarients from the previous exercise in time. 
\begin{align}
\int_0^t \der{}{s}\underbrace{\int_\R u\dd x}_{F(s)}\dd s &= 0\\
\iff F(t) &= F(0)\\
\iff \int_\R u(x,t)\dd x&=\int_\R u(x,0) \dd x
\end{align}
Where we simply used the fundamental theorem of calculus. So the integral over $u$ is bounded by the integral of the initial condition and one could write if wanted
\begin{align}
\int_\R u(x,t)\dd x\leq\int_\R u(x,0) \dd x
\end{align}
In the same manner
\begin{align}
\int_0^t \der{}{s}\int_\R u^2(x,s)\dd x\, \dd s& =0\\
\iff \int_\R u^2(x,t)\dd x &= \int_\R u^2(x,0)\dd x \\
\Rightarrow \int_\R u^2(x,t)\dd x &\leq \int_\R u^2(x,0)\dd x .
\end{align}
This result can immediately written down in terms of the $l_2$ norm
\begin{align}
\norm{u(x,t)}^2_{L_2}\leq \norm{u(x,0)}^2_{L_2}
\end{align}
And for the last invarient
\begin{align}
\int_0^t \der{}{s}\int_\R u^3(x,s)-\frac 12 u_x^2(x,s)\dd x\, \dd s& =0\\
\iff \int_\R u^3(x,t)-\frac 12 u_x^2(x,t)\dd x &= \int_\R u^3(x,0)-\frac 12 u_x^2(x,0)\dd x \\
\Rightarrow \int_\R u^3(x,t)-\frac 12 u_x^2(x,t)\dd x &\leq \int_\R u^3(x,0)-\frac 12 u_x^2(x,0)\dd x.
\end{align}