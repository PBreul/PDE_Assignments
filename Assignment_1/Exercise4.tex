\section{Exercise 4}
We have the following pde given
\begin{align}
u_t = u_{xxx} +6uu_x, \quad u=u(x,t) \label{eq:kdv}
\end{align}
which is known as the Korteweg-de Vries equation.
\paragraph{(a)}\label{ch:4a}
We want to find a symmetry transformation of the dependent and independent variables
\begin{align}
S:(t,x,u)\rightarrow(T,X,U) = (at,bt,cu),
\end{align}
with $a,b,c$ nonzero constants, such that \cref{eq:kdv} takes the form
\begin{align}
U_T = U_{XXX} +6UU_X, \quad U=U(X,T).\label{eq:transformed_kdv}
\end{align}
Before starting the calculations, we note the chain rule of partial derivatives 
\begin{align}
\pder{f(x(t))}{t} = \pder{x(t)}{t}\pder{f(x)}{x}.\label{eq:chain_rule}
\end{align}
We calculate the individual terms in \cref{eq:transformed_kdv}
\begin{align}
U_T(X,T)&=\pder{}{T}U=c\pder{}{T}u(x,t) \stackrel{\cref{eq:chain_rule}}{=} c \underbrace{\pder{t}{T}}_{1/a}\pder{u(x,t)}{t}=\frac ca u_t,\\
\intertext{In the same manner we calculate}
U_X&= c\frac{\partial x}{\partial X} \frac{\partial}{\partial x} u(x,t)=\frac{c}{b} u_{x},\\
U_{XXX}&=c \frac{\partial^3 x}{\partial X^3} \frac{\partial^3}{\partial x^3} u(x,t)=\frac{c}{b^3} u_{xxx}.
\end{align}
Plugging this into \cref{eq:transformed_kdv} we get
\begin{flalign}
 &&\frac ca u_t &= \frac{c}{b^3} u_{xxx} +6\frac{c^2}{b}uu_x,&&\\
\iff &&   u_t &= \frac{a}{b^3} u_{xxx} +6\frac{c}{b}uu_x.&&
\end{flalign}
So we get the conditions
\begin{flalign}
&&\frac{a}{b^3}=1, &\qquad \frac{ac}{b}=1,&&\\
\iff&& a=b^3,& \qquad c=\frac{1}{b^2}.&&
\end{flalign}
We choose $b=\epsilon$ with $\epsilon$ non zero and find
\begin{align}
X = \epsilon x, \quad T = \epsilon^3 t,\quad U = \frac{1}{\epsilon^2} u. 
\end{align}
Now we assume, that $u(x,t) = f(x,t)$ is a solution to \cref{eq:kdv}. The found symmetries leave the equation invariant. So we can construct a second solution
\begin{align}
u(x,t) = \epsilon^2 f(X,T) =  \epsilon^2 f(\epsilon x, \epsilon^3 t).
\end{align}

\paragraph{(b)}
We propose a solution of the type 
\begin{align}
u(x,t)=f(\xi),\quad \xi= x-vt.
\end{align}
We insert this ansatz into\cref{eq:kdv}. With the chain rule \cref{eq:chain_rule} we calculate the individual terms
\begin{align}
u_t &= \partial_\xi u \underbrace{\partial_t \xi}_{-v}=-v \partial_\xi u,\\
u_x &= \partial_\xi u \partial_x \xi = \partial_\xi u ,\\
\Rightarrow u _{xxx}&= \partial_{\xi}^3 u.
\end{align}
The equation becomes 
\begin{align}
-v u_\xi = u_{\xi\xi\xi} + 6 u u_\xi.
\end{align}
Since $\xi$ is the only variable, we have an ODE, we will use $(\cdot)'$ as notation for derivative to make clear, we are dealing with an ODE
\begin{align}
-v u' = u''' +6 uu'.\label{eq:4b_ode}
\end{align}
We take the integral
\begin{align}
\int -v u' \dd\xi &= \int u''' +3(u^2)' \dd\xi\\
\Rightarrow v u+u''+3u^2 &= A.
\end{align}
With $A$ being some integration constant.  We multiply by $u'$ and integrate again
\begin{align}
\int vuu' +u'u'' + 3u'u^2 \dd\xi&= \int Au' \dd\xi\\
\Rightarrow \frac 12 vu^2 + \frac 12 (u')^2  + u^3&= Au +B
\end{align}
The Weierstrass $\mathscr{P}$-function solves equations of the form
\begin{align}
(\mathscr{P}')^2 = 4\mathscr{P}^3-g_1\mathscr{P}-g_2
\end{align}
We have to make a substitution, so we get rid of the quadradic term in $\mathscr{P}$.
We propose $u(\xi)=-2y(\xi)-\frac v6$. Inserting, gives (addition of cubic and quadratic term done with Wolframalpha)
\begin{align}
-8y^3+\frac{v^2y}{2}+\frac{v^3}{108} +2(y')^2&=A\left(-2y-\frac v6\right) +B\\
\iff (y')^2 &= 4y^3-\underbrace{\left(\frac{v^2}{4}+A\right)}_{g_1}y - \underbrace{\left(\frac{v^3}{216} +\frac{Av}{12} - \frac B2\right)}_{g_2}
\end{align}
So the solution to \cref{eq:kdv} is given by
\begin{align}
u(x,t) &= -2\mathscr{P}(\xi,g_1,g_2) - \frac v6, \label{eq:sol_4b}\\
\xi = x-vt,& \quad g_1=\frac{v^2}{4}+A, \quad g_2=\frac{v^3}{216} +\frac{Av}{12} - \frac B2.
\end{align}
\paragraph{(c)}
We know, from exercise (a) that \cref{eq:sol_4b} will also be solved by
\begin{align}
u(x,t)=\epsilon^2\left(-2\mathscr{P}(\epsilon\left( x-\epsilon^2v t\right), g_1,g_2)-\frac{v}{6}\right).
\end{align}
For larger $\epsilon$ the amplitude of $u$ will clearly be larger. But at the same time, the time evolution will be faster, because we have an additional $\epsilon^2$ in front of $t$, compared to $x$. Or we could say, that we absorb the $\epsilon^2$ into a new velocity $\tilde{v}=v\epsilon^2$ which will of course be larger for larger $\epsilon$. So the wave propagates faster. \todo{Explanation}